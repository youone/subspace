%%! Author = johan
%%! Date = 2020-03-31

% Preamble
\documentclass[11pt]{article}
\title{Array Subspace Methods}
\author{Johan Lundqvist}
\date{March 2020}

% Packages
\usepackage{graphicx}
\usepackage{bm}
\usepackage{amssymb,amsmath}

\setlength{\parindent}{0em}
\setlength{\parskip}{0.7em}

\renewcommand{\vct}[1]{\boldsymbol{#1}}
\renewcommand{\dotprod}[2]{\vct{#1}\cdot\vct{#2}}

% Document
\begin{document}

\maketitle

\section{Plane Wave}
\label{sec:plane-wave}
Consider a plane wave in 3-dimensional space with wave number $\vct{k}(\vct{\xi})$.
It is here parametrized by $p$ variables contained in $\vct{\xi}$ (e.g. $p=2$, azimuth and elevation, $\vct{\xi}=[\varphi, \theta]$).

In a point selected as $\vct{r}=\vct{0}$ it will give a signal $s(t)\sim e^{-i\omega t+\phi}$, where $\omega=c/k$ and the phase $\phi$ will depend on the location of this origo.

Assuming a stationary frequency $\omega$ the signal $x(t)$ at an arbitrary point $\vct{r}$ will be given by a spatial phase shift of the amount ${\vct{k}(\vct{\xi})}\cdot{\vct{r}}$.
Hence
\begin{equation}
    x(\vct{r},t) = s(t)e^{i{\vct{k}(\vct{\xi})}\cdot{\vct{r}}}
\end{equation}

\section{Antenna Array with a Single Source}
\label{sec:antenna-array-with-a-single-source}
~\cite{Paulraj2016SubspaceMF,GonenMendel}
Let there be $n$ isotropically receiving antennas in the system at positions $\vct{r}_1, \vct{r}_2, \dots, \vct{r}_n$.
In the model presented here, signals collected at these positions are
\begin{equation}
    \begin{split}
        x_1(t)& = s(t)e^{i{\vct{k}(\vct{\xi})}\cdot{\vct{r}_1}} + \vct{n}_1(t)\\
        &\vdots\\
        x_n(t)& = s(t)e^{i{\vct{k}(\vct{\xi})}\cdot{\vct{r}_n}} + \vct{n}_n(t)
    \end{split}
\end{equation}

where $\vct{n}_k(t)$ denotes additive white Gaussian noise for channel $k$.
In matrix notation we have
\begin{equation} \label{eq:1}
\begin{split}
    \begin{bmatrix}
        x_1(t)\\
        \vdots\\
        x_n(t)
    \end{bmatrix}
    =&
    \begin{bmatrix}
        e^{i{\vct{k}(\vct{\xi})}\cdot{\vct{r}_1}}\\
        \vdots\\
        e^{i{\vct{k}(\vct{\xi})}\cdot{\vct{r}_n}} \\
    \end{bmatrix}
    s(t) +
    \begin{bmatrix}
        n_1(t)\\
        \vdots\\
        n_n(t)
    \end{bmatrix}
    %\end{equation}
    %\begin{equation}
    \\
    \vct{x}(t) =& \vct{a}(\vct{\xi})s(t) + \vct{n}(t)\\
\end{split}
\end{equation}

Already at this point some observations of fundamental importance can be made. The first being that the signal sampled at time $t$, $\vct{x}(t)$, will be a scalar multiple of the vector $\vct{a}(\vct{\xi})$ plus a, hopefully small, addition from the noise term $\vct{n}(t)$. The scalar factor is $s(t)$ and the vectors belong to $\mathbb C^n$. Figure \ref{fig:subspace1dim} illustrates this picture.

The sub-space spanned by $\vct{a}(\vct{\xi})$ (in this case 1-dimensional) is called the \textit{column space} of the $n\times1$ matrix $\vct{a}(\vct{\xi})$, but is referred to as the \textit{signal space} in array theory. The vector $\vct{a}(\vct{\xi})$ is called the \textit{steering vector} of the array, and besides its $\vct{\xi}$-dependence (e.g. the direction of the signal source) it is only affected by the spatial array configuration in this simple example.

In a hypothetical noise free system all measured signal samples will lie in the signal space. Due to noise however, the measurements will have an additional random component that lies in a sub-space orthogonal to the signal space, called the \textit{noise}- or \textit{null-space}.

The set of all possible values that $\vct{a}(\vct{\xi})$ can assume is called the \textit{array manifold}. As it only depends on $\vct{\xi}$ it will constitute a $p$-dimensional surface in $\mathbb C^n$. Note that the measured signal $\vct{x}(t)$ for a specific plane wave cannot be a scalar multiple of any possible $\vct{a}(\vct{\xi})$, but only the one given by the specific parameters $\vct{\xi}$ for that wave.

A straight forward procedure for estimating the correct value of $\vct{\xi}$ (we can call it $\hat{\vct{\xi}}$) is to solve the linear system \eqref{eq:1} for $s(t)$, and then scan the $\vct{\xi}$-parameter space in order to find the minimum of the difference between the measured value $x(t)$ and $\vct{a}(\vct{\xi})\hat{s}(t)$, where $\hat{s}(t)$ is the solution to \eqref{eq:1}. That is
\begin{equation}\label{eq:2}
\hat{\vct{\xi}}=\underset{\vct{\xi}}{\mathrm{arg\ min}}\ \|\vct{x}(t) - \vct{a}(\vct{\xi})\hat{s}(t)\|^2
\end{equation}
\\
%Having found $\hat{s}(t)$, looking at figure \ref{fig:subspace1dim}, this problem looks trivial.
Now, if $\vct{a}$ were to be an invertible $n\times n$ matrix\footnotemark the unique solution would be $\hat{s}(t)=\vct{a}^{-1}\vct{x}(t)$. In this example however $\vct{a}$ is a $1\times n$ matrix and therefor in general describes an over determined equation system. A standard way of solving such a system is to multiply by $\vct{a}^*$ and instead solving the system ($\vct{a}^*\vct{a}$ is real, square and symmetric)
\footnotetext{This would be the case with $n$ plane waves as will be described later}

\begin{figure}[t]
    \def\svgwidth{250}
    \input{subspace1dim.pdf_tex}
    \centering
    \caption{\textit{Array sub spaces}}
    \label{fig:subspace1dim}
\end{figure}


\begin{equation}
    \vct{a}^*\vct{a}s(t) = \vct{a}^*\vct{x}(t)
\end{equation}
\\
Hence
\begin{equation}
    \hat{s}(t) = (\vct{a}^*\vct{a})^{-1}\vct{a}^*\vct{x}(t)
\end{equation}

In this case $\vct{a}$ has only one column and $\vct{a}^*\vct{a}$ is just scalar norm of $\vct{a}$ (which in fact is a real $1\times 1$ matrix) that and can be divided by. Multiplying both sides by $\vct{a}$ then gives

\begin{equation}\label{eq:3}
\\\vct{a}\hat{s}(t) = \vct{a}\frac{\vct{a}^*\vct{x}(t)}{\vct{a}^*\vct{a}}
\end{equation}

With a little effort the expression on the right is seen to be the the projection of $\vct{x}(t)$ on $\vct{a}(\vct{\xi})$.
In fact the operator $\vct{p}_{\vct{a}}=\vct{a}(\vct{a}^*\vct{a})^{-1}\vct{a}^*$ is called a projection operator.
Plugging ~\eqref{eq:3} into ~\eqref{eq:2} the procedure then becomes to look for values of $\vct{\xi}$ that minimizes the expression
\begin{equation}
    \|\vct{x}(t)-\vct{p}_{\vct{a}}(\vct{\xi})\vct{x}(t)\|^2
\end{equation}
That is, find the value of $\vct{\xi}$ that makes the projection of the measurement $\vct{x}(t)$ on $\vct{a}(\vct{\xi})$ lie as close as possible to $\vct{x}(t)$ itself.

\subsection{Beamforming}

\subsection{Statistics}

\subsection{Diagonalization}

\section{Multiple Sources}

With $d$ plane waves parametrized by $\vct{\xi}_1 \dots\ \vct{\xi}_d$ entering the antenna field, superposition gives the measured signals
\begin{equation}
    \begin{split}
        x_1(t)& = s_1(t)e^{i{\vct{k}(\vct{\xi}_1)}\cdot{\vct{r}_1}} + \dots + s_d(t)e^{i{\vct{k}(\vct{\xi}_d)}\cdot{\vct{r}_1}}\\
        &\vdots\\
        x_n(t)& = s_1(t)e^{i{\vct{k}(\vct{\xi}_1)}\cdot{\vct{r}_n}} + \dots + s_d(t)e^{i{\vct{k}(\vct{\xi}_d)}\cdot{\vct{r}_n}}\\
    \end{split}
\end{equation}
\\and in matrix notation
\begin{equation} \label{eq:4}
\begin{split}
    \begin{bmatrix}
        x_1(t)\\
        \vdots\\
        x_n(t)
    \end{bmatrix}
    =&
    \begin{bmatrix}
        e^{i{\vct{k}(\vct{\xi}_1)}\cdot{\vct{r}_1}} & \dots & e^{i{\vct{k}(\vct{\xi}_d)}\cdot{\vct{r}_1}}\\
        \vdots & \ddots & \vdots\\
        e^{i{\vct{k}(\vct{\xi}_1)}\cdot{\vct{r}_n}} & \dots & e^{i{\vct{k}(\vct{\xi}_d)}\cdot{\vct{r}_n}}\\
    \end{bmatrix}
    \begin{bmatrix}
        s_1(t)\\
        \vdots\\
        s_d(t)
    \end{bmatrix}
    \\
    %\end{equation}
    %\begin{equation}
    \vct{x}(t) =& \vct{A}(\vct{\xi})\vct{s}(t)\\
\end{split}
\end{equation}

where the notation for the full steering vector (now a $n\times d$ matrix) is changed to $\vct{A}(\vct{\xi})$ and the individual columns from now on are called $\vct{a}(\vct{\xi}_k)$.
\begin{equation}
    \vct{A}(\vct{\xi}) = \begin{bmatrix}\vct{a}(\vct{\xi}_1)&\dots&\vct{a}(\vct{\xi}_d)\end{bmatrix}
\end{equation}

Column wise matrix multiplication gives the measured signal vector as a linear combination of the columns of $\vct{A}(\vct{\xi})$ according to

\begin{equation} \label{eq:5}
\vct{x}(t) = s_1(t)\vct{a}(\vct{\xi}_1) + \dots + s_d(t)\vct{a}(\vct{\xi_d})
\end{equation}

And just like in the case with a single source this means that the signal is confined to a subspace of $\mathbb C^n$ spanned by the columns of $\vct{A}(\vct{\xi})$.


%Consider the following vectors with entries in $\R$: $\mathbf
%x=\vector{6}$, $\mathbf u=\avector{u}{7}$, $\mathbf a=\aDEFvector{8}$,
%$\mathbf w=\aDEFvector[w]{n+1}$.

%\citep{adams1995hitchhiker}

%$\vct{\xi}$
%$\vct{a}(\vct{\xi})$
%$\vct{x}(t)$


\bibliography{main}
\bibliographystyle{plain}

\end{document}

%\documentclass{article}
%\usepackage[utf8]{inputenc}
%
%\title{Array Subspace Methods}
%\author{Johan Lundqvist}
%\date{March 2020}
%
%\usepackage{graphicx}
%\usepackage{bm}
%\usepackage{amssymb,amsmath}
%%\usepackage{natbib}
%%\usepackage[sorting=none]{biblatex}
%%\usepackage[backend=bibtex,style=numeric]{biblatex}
%%\addbibresource{references.bib}
%%\bibliography{references}
%%\bibliographystyle{numeric}
%
%\setlength{\parindent}{0em}
%\setlength{\parskip}{0.7em}
%
%\newcommand{\R}{{\mathbb R}}
%\renewcommand{\vector}[1]{(x_1,x_2,\ldots,x_{#1})}
%\renewcommand{\vct}[1]{\boldsymbol{#1}}
%\renewcommand{\dotprod}[2]{\vct{#1}\cdot\vct{#2}}
%\newcommand{\avector}[2]{(#1_1,#1_2,\ldots,#1_{#2})}
%\newcommand{\aDEFvector}[2][a]{(#1_1,#1_2,\ldots,#1_{#2})}
%\DeclareMathOperator*{\argminA}{arg\,min}
%
%\begin{document}
%
%    \maketitle
%
%    \section{Plane Wave}
%    Consider a plane wave in 3-dimensional space with wave number $\vct{k}(\vct{\xi})$. It is here parametrized by $p$ variables contained in $\vct{\xi}$ (e.g. $p=2$, azimuth and elevation, $\vct{\xi}=[\varphi, \theta]$). In a point selected as $\vct{r}=\vct{0}$ it will give a signal $s(t)\sim e^{-i\omega t+\phi}$, where $\omega=c/k$ and the phase $\phi$ will depend on the location of this origo.
%
%    Assuming a stationary frequency $\omega$ the signal $x(t)$ at an arbitrary point $\vct{r}$ will be given by a spatial phase shift of the amount ${\vct{k}(\vct{\xi})}\cdot{\vct{r}}$. Hence
%
%    \begin{equation}
%        x(\vct{r},t) = s(t)e^{i{\vct{k}(\vct{\xi})}\cdot{\vct{r}}}
%    \end{equation}
%
%    \section{Antenna Array with a Single Source}
%    \cite{Paulraj2016SubspaceMF}
%    \cite{}
%    Let there be $n$ isotropically receiving antennas in the system at positions $\vct{r}_1, \vct{r}_2, ..., \vct{r}_n$. In the model presented here, signals collected at these positions are
%    \begin{equation}
%        \begin{split}
%            x_1(t)& = s(t)e^{i{\vct{k}(\vct{\xi})}\cdot{\vct{r}_1}} + \vct{n}_1(t)\\
%            &\vdots\\
%            x_n(t)& = s(t)e^{i{\vct{k}(\vct{\xi})}\cdot{\vct{r}_n}} + \vct{n}_n(t)
%        \end{split}
%    \end{equation}
%    where $\vct{n}_k(t)$ denotes additive white Gaussian noise for channel $k$. In matrix notation we have
%    \begin{equation} \label{eq:1}
%    \begin{split}
%        \begin{bmatrix}
%            x_1(t)\\
%            \vdots\\
%            x_n(t)
%        \end{bmatrix}
%        =&
%        \begin{bmatrix}
%            e^{i{\vct{k}(\vct{\xi})}\cdot{\vct{r}_1}}\\
%            \vdots\\
%            e^{i{\vct{k}(\vct{\xi})}\cdot{\vct{r}_n}} \\
%        \end{bmatrix}
%        s(t) +
%        \begin{bmatrix}
%            n_1(t)\\
%            \vdots\\
%            n_n(t)
%        \end{bmatrix}
%        %\end{equation}
%        %\begin{equation}
%        \\
%        \vct{x}(t) =& \vct{a}(\vct{\xi})s(t) + \vct{n}(t)\\
%    \end{split}
%    \end{equation}
%    \\
%
%    Already at this point some observations of fundamental importance can be made. The first being that the signal sampled at time $t$, $\vct{x}(t)$, will be a scalar multiple of the vector $\vct{a}(\vct{\xi})$ plus a, hopefully small, addition from the noise term $\vct{n}(t)$. The scalar factor is $s(t)$ and the vectors belong to $\mathbb C^n$. Figure \ref{fig:subspace1dim} illustrates this picture.
%
%    The sub-space spanned by $\vct{a}(\vct{\xi})$ (in this case 1-dimensional) is called the \textit{column space} of the $n\times1$ matrix $\vct{a}(\vct{\xi})$, but is referred to as the \textit{signal space} in array theory. The vector $\vct{a}(\vct{\xi})$ is called the \textit{steering vector} of the array, and besides its $\vct{\xi}$-dependence (e.g. the direction of the signal source) it is only affected by the spatial array configuration in this simple example.
%
%    In a hypothetical noise free system all measured signal samples will lie in the signal space. Due to noise however, the measurements will have an additional random component that lies in a sub-space orthogonal to the signal space, called the \textit{noise}- or \textit{null-space}.
%
%    The set of all possible values that $\vct{a}(\vct{\xi})$ can assume is called the \textit{array manifold}. As it only depends on $\vct{\xi}$ it will constitute a $p$-dimensional surface in $\mathbb C^n$. Note that the measured signal $\vct{x}(t)$ for a specific plane wave cannot be a scalar multiple of any possible $\vct{a}(\vct{\xi})$, but only the one given by the specific parameters $\vct{\xi}$ for that wave.
%
%    A straight forward procedure for estimating the correct value of $\vct{\xi}$ (we can call it $\hat{\vct{\xi}}$) is to solve the linear system \eqref{eq:1} for $s(t)$, and then scan the $\vct{\xi}$-parameter space in order to find the minimum of the difference between the measured value $x(t)$ and $\vct{a}(\vct{\xi})\hat{s}(t)$, where $\hat{s}(t)$ is the solution to \eqref{eq:1}. That is
%
%    \begin{equation}\label{eq:2}
%    \hat{\vct{\xi}}=\underset{\vct{\xi}}{\mathrm{arg\ min}}\ \|\vct{x}(t) - \vct{a}(\vct{\xi})\hat{s}(t)\|^2
%    \end{equation}
%    \\
%    %Having found $\hat{s}(t)$, looking at figure \ref{fig:subspace1dim}, this problem looks trivial.
%    Now, if $\vct{a}$ were to be an invertible $n\times n$ matrix\footnotemark the unique solution would be $\hat{s}(t)=\vct{a}^{-1}\vct{x}(t)$. In this example however $\vct{a}$ is a $1\times n$ matrix and therefor in general describes an over determined equation system. A standard way of solving such a system is to multiply by $\vct{a}^*$ and instead solving the system ($\vct{a}^*\vct{a}$ is real, square and symmetric)
%
%    \footnotetext{This would be the case with $n$ plane waves as will be described later}
%
%    \begin{figure}[t]
%        \def\svgwidth{250}
%        \input{subspace1dim.pdf_tex}
%        \centering
%        \caption{\textit{Array sub spaces}}
%        \label{fig:subspace1dim}
%    \end{figure}
%
%
%    \begin{equation}
%        \vct{a}^*\vct{a}s(t) = \vct{a}^*\vct{x}(t)
%    \end{equation}
%    \\
%    Hence
%    \begin{equation}
%        \hat{s}(t) = (\vct{a}^*\vct{a})^{-1}\vct{a}^*\vct{x}(t)
%    \end{equation}
%
%    In this case $\vct{a}$ has only one column and $\vct{a}^*\vct{a}$ is just scalar norm of $\vct{a}$ (which in fact is a real $1\times 1$ matrix) that and can be divided by. Multiplying both sides by $\vct{a}$ then gives
%    \begin{equation}\label{eq:3}
%    \\\vct{a}\hat{s}(t) = \vct{a}\frac{\vct{a}^*\vct{x}(t)}{\vct{a}^*\vct{a}}
%    \end{equation}
%    \\
%    With a little effort the expression on the right is seen to be the the projection of $\vct{x}(t)$ on $\vct{a}(\vct{\xi})$. In fact the operator $\vct{p}_{\vct{a}}=\vct{a}(\vct{a}^*\vct{a})^{-1}\vct{a}^*$ is called a projection operator. Plugging \eqref{eq:3} into \eqref{eq:2} the procedure then becomes to look for values of $\vct{\xi}$ that minimizes the expression
%    \\
%    \begin{equation}
%        \|\vct{x}(t)-\vct{p}_{\vct{a}}(\vct{\xi})\vct{x}(t)\|^2
%    \end{equation}
%    \\
%    That is, find the value of $\vct{\xi}$ that makes the projection of the measurement $\vct{x}(t)$ on $\vct{a}(\vct{\xi})$ lie as close as possible to $\vct{x}(t)$ itself.
%
%    \subsection{Beamforming}
%
%    \subsection{Statistics}
%
%    \subsection{Diagonalization}
%
%    \section{Multiple Sources}
%
%    With $d$ plane waves parametrized by $\vct{\xi}_1 \dots\ \vct{\xi}_d$ entering the antenna field, superposition gives the measured signals
%    \begin{equation}
%        \begin{split}
%            x_1(t)& = s_1(t)e^{i{\vct{k}(\vct{\xi}_1)}\cdot{\vct{r}_1}} + ... + s_d(t)e^{i{\vct{k}(\vct{\xi}_d)}\cdot{\vct{r}_1}}\\
%            &\vdots\\
%            x_n(t)& = s_1(t)e^{i{\vct{k}(\vct{\xi}_1)}\cdot{\vct{r}_n}} + ... + s_d(t)e^{i{\vct{k}(\vct{\xi}_d)}\cdot{\vct{r}_n}}\\
%        \end{split}
%    \end{equation}
%    \\and in matrix notation
%    \begin{equation} \label{eq:4}
%    \begin{split}
%        \begin{bmatrix}
%            x_1(t)\\
%            \vdots\\
%            x_n(t)
%        \end{bmatrix}
%        =&
%        \begin{bmatrix}
%            e^{i{\vct{k}(\vct{\xi}_1)}\cdot{\vct{r}_1}} & \dots & e^{i{\vct{k}(\vct{\xi}_d)}\cdot{\vct{r}_1}}\\
%            \vdots & \ddots & \vdots\\
%            e^{i{\vct{k}(\vct{\xi}_1)}\cdot{\vct{r}_n}} & \dots & e^{i{\vct{k}(\vct{\xi}_d)}\cdot{\vct{r}_n}}\\
%        \end{bmatrix}
%        \begin{bmatrix}
%            s_1(t)\\
%            \vdots\\
%            s_d(t)
%        \end{bmatrix}
%        \\
%        %\end{equation}
%        %\begin{equation}
%        \vct{x}(t) =& \vct{A}(\vct{\xi})\vct{s}(t)\\
%    \end{split}
%    \end{equation}
%
%    where the notation for the full steering vector (now a $n\times d$ matrix) is changed to $\vct{A}(\vct{\xi})$ and the individual columns from now on are called $\vct{a}(\vct{\xi}_k)$.
%    \begin{equation}
%        \vct{A}(\vct{\xi}) = \begin{bmatrix}\vct{a}(\vct{\xi}_1)&\dots&\vct{a}(\vct{\xi}_d)\end{bmatrix}
%    \end{equation}
%
%    Column wise matrix multiplication gives the measured signal vector as a linear combination of the columns of $\vct{A}(\vct{\xi})$ according to
%
%    \begin{equation} \label{eq:5}
%    \vct{x}(t) = s_1(t)\vct{a}(\vct{\xi}_1) + \dots + s_d(t)\vct{a}(\vct{\xi_d})
%    \end{equation}
%
%    And just like in the case with a single source this means that the signal is confined to a subspace of $\mathbb C^n$ spanned by the columns of $\vct{A}(\vct{\xi})$.
%
%
%    %Consider the following vectors with entries in $\R$: $\mathbf
%    %x=\vector{6}$, $\mathbf u=\avector{u}{7}$, $\mathbf a=\aDEFvector{8}$,
%    %$\mathbf w=\aDEFvector[w]{n+1}$.
%
%    %\citep{adams1995hitchhiker}
%
%    %$\vct{\xi}$
%    %$\vct{a}(\vct{\xi})$
%    %$\vct{x}(t)$
%
%    \bibliography{main}
%    \bibliographystyle{plain}
%
%\end{document}
